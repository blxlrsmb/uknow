\section{Test Procedure}
\label{sec:test_procedure}
  \subsection{Testers}
    zxytim, ppwwyyxx, vuryleo, jiakai
  \subsection{Period}
    18:00 Dec 13th, 2013 to 18:00 Dec 14th, 2013.
  \subsection{Functional Tests}
\begin{enumerate}

    \item Test Case Designing
      First, read the API list, select out APIs that need to be tested.
      Then, design use cases for each API based on its function.
      At last, translate test cases into proper request format and add them into the test suite.

      Several cases that designed to test the robustness of the APIs are recorded as well.
      These cases will perform illegal requests such as requests with wrong format, permission overstep and injection.
    \item Environment Setting
      At first, use virtualenv to build a virtual python environment for this project.
      Then use install tool located in manage suite to set up runtime environment for it.

      In order to provide production environment, a mongodb server and a redis server is also required.
    \item Software Configuring
      In this test, the default configure is applied. So no interfere is needed in this step.
    \item Test Configuring
      In order to test some third party function, correct authentication information is required.
      The configure file can be found easily in test suite.
    \item Test Executing
      Execute test entry scripts to do the tests.
    \item Log Analyzing
      After getting the test log, parse out the useful information and ordinate them into proper format.

\end{enumerate}
  \subsection{Performance Tests}
    \subsubsection{Test Case Designing}
      Several APIs that may have heavy loads are picked out to get the worst performance data.
      Then a random set of APIs is generated in order to test the average performance.
      Besides, some APIs that may be requested most frequently are selected to get the common performance.

      After all, translate the test requests in functional test into well formed Internet requests
      and put them in proper places.
    \subsubsection{Environment Setting}
      In propose of simulating real world performance,
      client and server are located on different machines, so the network transform is considered.

      To do so, deploy the project in production mode on one machine, and get the test suite on another.
    \subsubsection{Test Executing}
      Execute test entry scripts to do the tests.
    \subsubsection{Log Analyzing}
      Performance test will generate plenty of log.
      So a data visualization procedure is taken to display the data in different collections and patterns.
