\newcolumntype{L}{>{\centering\arraybackslash}m{3cm}}

\section{Introduction}
\label{sec:intro}
\subsection{Software Introduction}
\begin{center}
\begin{table}[!ht]
	\begin{tabular}{|c|c|c|c|c|}
		\hline
		Component & Language & Environment & Lines & Remark \\\hline
        fetcher & Python2 & POSIX & 747 & Including general and user ones\\ \hline
        prefilter & Python2 & POSIX & 54 & Not including Training material \\ \hline
        api-website & Python2 & POSIX & 426 & Not including execute scripts\\ \hline
        frontend-page & HTML/JS/CSS & POSIX & 558 & CSS:212 JavaScript:153 HTML:193\\ \hline
        manage & Python2/Shell etc & POSIX & 117 & Management scripts \\ \hline
        library & Python2 & POSIX & 699 & Common library for all components \\ \hline
        tests & Python2/Shell & POSIX & 336 & Including unit test and benchmark \\ \hline
        sum &  & POSIX & 2437 & \\ \hline
	\end{tabular}
\end{table}
\end{center}

\subsection{Document Introduction}
\label{sec:introduction}
	In the following sections of this document, we will present test scheme and
	results in detail.

	In~\secref{test_scheme}, the methodology and environment the test is conducted
	is described. Detailed and accurate test procedure will be present in~\secref{test_procedure}.
	Test result will be covered in~\secref{test_result}. At last, evaluation of the test
	will be discussed in~\secref{test_evaluation}.

