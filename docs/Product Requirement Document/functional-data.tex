%File: functional-data.tex
%Date: Sat Oct 19 13:49:50 2013 +0800
%Author: Yuxin Wu <ppwwyyxxc@gmail.com>

\subsection{Data Management}

\subsubsection{Items}

An item is piece of information retrieved from various source, and
showed up in lists for further reading. It comprise of following parts:

\begin{itemize}
\itemsep1pt\parskip0pt\parsep0pt
\item
  Title
\item
  Source
\item
  Content
\item
  Labels
\item
  Comments
\end{itemize}

It is the core entity in the system. A user can manipulate items in
following ways:

\begin{description}
\item[Removal] \hfill

  When user feels unpleasant or tired of specific item, he/she
  may remove this item using button provided. Removal operation can be
  reversed whinin a short period of time to prevent from mis-clicking
  the button.
\item[Archive] \hfill

  User can store specific item for further reading or recap in
  the future. Archived items still show up in item flow, and will be
  labeled as `archived'.
\item[Share] \hfill

  If an item fascinates a user, he may share it among popular SNS.
  When first sharing an item, user will authenticate its account on
  desired SNS website, and after confirmation, a sharing information
  will be posted on choosed SNS.
\item[Like] \hfill

  User can show his/her opinion on specific item, and others will
  know the statistics, and could be used as an score to evaluate an
  item. High score items are more likely to rank higher than low score
  items.
\item[Comment] \hfill

  User can post comments under an item. Posts will be stored in
  this system, and can optionally be posted synchronously on item
  source, if the source website supports such function.

  Comments is organized hierarchically. If a user is intend to reply
  another one's comment, he/she can click the reply button, and the
  reply form will show just below the comment.

  When complete typing, click the reply button and the comments will be
  posted, and the form will disappear as you finish commenting.
\end{description}

Further more, system can learn from user behavious on items and to
recommend items related to users interest.

\subsubsection{Labels}

A label is an attribute describing items. An item can have multiple
label. A label associated with an item may be automatically assigned by
system, and tagged by user. A label can be either a system-wide label,
which is visible to all user, or a user specified label which indicates
user preference on an item.

Label may come from:

\begin{description}
\item[System Pre-tagging]
  An item may arrive to a user with pre-tagged
  labels. These pre-tagged labels are vital to subsequent data
  processing, such as tag-filtering plugin in tab.
\item[User-tagging]
  Same item could have distinct meaning to different user.
  User can tag item with labels cater to their taste, as well as remove
  labels that could lead to misunderstanding to itself.
\end{description}

As decribed aforementioned, functions like `archive' is actually tag a
item which label `archived'

\subsubsection{Plugins}

Plugin is a essential concept in the system, which comprises the
implementations of varied functionalities in the system. A plugin should
process a bunch of items, returning processed items. The number of items
before and after need not to be the same, that is, a plugin can either
shrink items or enrich items (filtering job) or process on contents of
items.

User can choose a subset of provided plugins to employ on its items.
Plugin can be either scoped to a `tab' or can be applied system-wide.

Plugins can be configurable, but configuration is plugin-dependent.

Examples of plugins:

\begin{itemize}
\itemsep1pt\parskip0pt\parsep0pt
\item
  Tag-filtering
\item
  Highlight
\item
  Emotion tagging
\item
  Face detection in image
\item
  Related items
\item
  etc.
\end{itemize}

\subsubsection{Tabs}

Tab is a collection of filtered items, in which the filter is defined by
plugins. A typical use of tab is to divide items into different
categories, which may be exclusive or not.

Tabs can be added or removed on-the-fly with `add' or `remove' button,
add define its behaviour with plugins.

Moreover, tab layout, item flow style, etc., are also configurable via
tab configuration.
