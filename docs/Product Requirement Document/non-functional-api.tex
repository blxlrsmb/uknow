\subsection{Web API}

\label{sec:webapi}
Since we provide several different clients, the communication between
client and server should be universal. Thus, web API implemented in Python and based on JSON will
be used.

JSON (JavaScript Object Notation)\footnote{\url{http://www.json.org/}} is a lightweight data-interchange
format. It is easy for humans to read and write. It is easy for machines
to parse and generate. It is based on a subset of the JavaScript
Programming Language, Standard ECMA-262 3rd Edition - December 1999\footnote{\url{http://www.ecma-international.org/publications/files/ecma-st/ECMA-262.pdf}}.

\subsubsection{Basic Structure}

All returned valued is a dict containing some specific keys. If an error
occurred during an API access, a key named \texttt{error} will be
provided with a key named \texttt{msg} showing the reason of error.
Otherwise, the desired data will be returned.

\subsubsection{User Management}

All user management related action should be accessible by API including
but not limited to:

\begin{itemize}
\itemsep1pt\parskip0pt\parsep0pt
\item
  Register an account
\item
  User login (validate)
\item
  Edit profile
\item Withdraw authority for specific third-party web service
\item
  Delete an account
\end{itemize}

When a password is sent, SSL/TLS protocal should be used.

\subsubsection{Data Flow}
Any client of Uknow shall follow an uniformed protocal to commuticate data with Uknow backend server.
Thus, Uknow server shall provide web API for clients to fetch feeds data, mainly consisting of the following parts:
\begin{enumerate}
  \item Get new feeds
  \item Tag an item
  \item Update tab configurations
  \item Update tab-level plugins
\end{enumerate}


