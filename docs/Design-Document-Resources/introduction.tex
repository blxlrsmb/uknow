%File: introduction.tex
%Date: Sat Oct 19 17:17:08 2013 +0800
%Author: Yuxin Wu <ppwwyyxxc@gmail.com>

\section{Introduction}
  \subsection{Purpose}
    For that there are many developers that got involved into this project,
    this document was wrote, in order to provide a mannual that developers can refer to
    when they are confused or forgot some protocol.
    Besides, this is also an introduction for new developers, who may join after this project was done.
    It will act as a overview of this project, to which new hands can refer.

  \subsection{Background}

    Uknow InfoHub is a collector that aimed to gather information.
    It's a project which was assigned as team homework of 2013\-2014 Fall, Softerware Engineering class of Tsinghua University.
    And it was bootstraped by blxlrsmb group, which has blahgeek, jiakai66, ppwwyyxx, vera, vuryleo and zxytim as members.
    It would be run on a VPS located in U.S., provided by Linode Ltd.\ co.

  \subsection{Definition}
    \subsubsection{Items}

      An item is piece of information retrieved from various sources, and
      shown in lists for further reading. It consists of following parts:

      \begin{itemize}
      \itemsep1pt\parskip0pt\parsep0pt
      \item
        Title
      \item
        Source
      \item
        Content
      \item
        Labels
      \item
        Comments
      \end{itemize}

      Item is the core entity in Uknow system.

    \subsubsection{Labels}

      A label is an attribute describing items. An item can have multiple
      labels. A label associated with an item may be automatically assigned by
      system, or tagged by users. A label can be either a system-wide label,
      which is visible to all user, or a user specified label which indicates
      user preferences on an item.

      As described aforementioned, functions like `archive' is actually a process of tagging an
      item by the label `archived'

    \subsubsection{Plugins}

      Plugin is an essential concept in Uknow system, which comprises the
      implementations of varied functionalities in the system. A plugin should
      process a bunch of items, returning processed items. The number of items
      before and after need not to be the same, that is, a plugin can either
      shrink items or enrich items (filtering job) or process on contents of
      items.

    \subsubsection{Tabs}

      Tab is a collection of filtered items, in which the filter is defined by
      plugins. A typical use of tab is to divide items into different
      categories, which can be exclusive or not.

